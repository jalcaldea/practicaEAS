\subsection{Perspectiva del producto}

VIRUTA debe ser implantado en un entorno de explotación específico y dependiente de otros sistemas ya existentes. Como ya se ha enunciado anteriormente, la aplicación deberá correr en dispositivos inteligentes.\\

Estos dispositivos, usarán Android 1.6 o superior, iOS o WindowsPhone como sistema operativo, por lo que estas serán las especificaciones mínimas de un dispositivo:

\begin{itemize}
\item HSDPA
\item Al menos 200MHz de procesador
\item 32 MB de memoria RAM
\item Procesador ARMv5 o superior 
\item Puerto USB
\item Bluetooth
\end{itemize}

Los dispositivos necesitarán deberán contar con una impresorora Bluetooth, que permitirá imprimir los billetes.\\

Además, la aplicacción móvil a desarrollar deberá interacturar con el sistema central (SC) de TRANSFER para la descarga de las operaciones de venta efectuadas. Este proceso se llevará a cabo mediante conexión permenete entre los dispositivos y el SC.\\

El sistema no deberá interactuar con el tren ni con ninguno de los sistemas que pudiera haber en él.\\

Los requisitos concretos para los interfaces del sistema serán enunciados en la sección 3 de este documento.


\subsection{Funciones del producto}

La funcionalidad del sistema puede descomponerse conceptualmente en los siguientes módulos:

\begin{enumerate}
\item Autenticación de usuarios
	\begin{enumerate}
	\item Autenticación del usuario
	\end{enumerate}
\item Venta de billetes
	\begin{enumerate}
	\item Venta del billete.
	\item Pago con tarjeta.
	\item Impresión del billete.
	\item Impresión del justificante de compra.
	\end{enumerate}
\item Descarga de operaciones
	\begin{enumerate}
	\item Generación del archivo intermedio de operaciones
	\end{enumerate}
\item Actualización del software
	\begin{enumerate}
	\item Actualización de tarifas.
	\item Actualización de descuentos.
	\item Actualización de la red ferroviaria.
	\item Actualización de usuarios.
	\end{enumerate}
\end{enumerate}

\subsection{Características del usuario}

El usuario estándar de VIRUTA será un revisor de TRANSFER. Los revisores de TRANSFER son personas de avanzada edad con poca exposición a la tecnología. Además, es posible que sufran problemas de visión.

\subsection{Restricciones generales}

El sistema tendrá unas mínimas garantías de seguridad. Solo podrá ser operado mediante previa autentificación del usuario en el sistema. Esta autenticación se realizará mediante el uso de un nombre de usuario y contraseña:\\

\textbf{Nombre de usuario:} Mínimo 4 caracteres alfanuméricos.\\

\textbf{Contraseña:} Mínimo 8 caracteres, incluyendo mayúsculas, minúsculas, números y al menos un carácter especial.

\subsection{Suposiciones y dependencias}

El presente documento se ha realizado con la información disponible en la fecha del mismo. Dada la volatilidad esperada de alguno de los requisitos enunciados, este documento queda sujeto a posibles actualizaciones.

\subsection{Requisitos futuros}

Es posible que el sistema evolucione en un futuro de tal forma que VIRUTA pueda interactuar con el directorio ligero de usuarios de TRANSFER, con el objetivo de unificar la gestión de usuarios de todos los sistemas de la compañía ferroviaria.\\